%--------------------BLOQUE DE FORMATO-------------
\usepackage[letterpaper,left=3cm,top=2.5cm,right=3cm,bottom=2.5cm]{geometry} %margenes
\usepackage{amsmath,amsthm,amsfonts,amssymb,amscd} 
\usepackage[utf8]{inputenc} %carácteres acentuados
\usepackage{enumerate}
\usepackage{mathrsfs}
\usepackage{xcolor}
\usepackage{graphicx} %Package to insert images and graphics
\usepackage{listings} %Package to listings enviornment e.g. MATLAB, Python, Fortran.
\usepackage{color}
\usepackage{bm}
\usepackage{nicefrac} %Para las fracciones
\usepackage{booktabs}
\usepackage{url}

\usepackage{parskip}
\setlength{\parindent}{0cm}

\usepackage[spanish,mexico]{babel} % Language hyphenation and typographical rules
%\usepackage[spanish,es-tabla]{babel}%cambia el idioma de "Table" a "Tabla"


%------------PARA AJUSTAR FUENTES-----------------------------
\usepackage[T1]{fontenc}%habilita caracteres no latinos
\usepackage[scaled]{helvet}%fuente arial=helvet
\usepackage{sfmath} %paquete para ajustar la fuente matematica a serif. 
%\usepackage{mathastext} %formulas matematicas, de acuerdo a la fuente elegida.
%%%%%%%%%PROCESAR CON XELATEX%%%%
%\usepackage{Arial}
%\usepackage{arevmath}
%%%%%%%%%%%%%%%%%%%%%%%%%%%%%%%%%
\renewcommand*\familydefault{\sfdefault}%fuente
\linespread{1.5} % Line spacing
%------------------------------------------------------------------------------

\usepackage{fancyhdr} % Headers and footers
\pagestyle{fancy} % All pages have headers and footers
\fancyhead{} % Blank out the default header
\fancyfoot{} % Blank out the default footer

\renewcommand{\headrulewidth}{0pt} %eliminar regla del encabezado
\fancyfoot[RO,RE]{\thepage} % Custom footer text

\usepackage{titlesec} %modifición de secciones
\titleformat{\section}[hang]{\normalfont\bfseries\centering}{\thesection}{5pt.}{} %modificación secciones
\titleformat{\subsection}[hang]{\normalfont\bfseries\centering}{\thesubsection}{5pt}{} %modificación subsecciones
\titleformat{\subsubsection}[hang]{\normalfont\bfseries}{\thesubsubsection}{5pt}{} %modificación subsecciones

\usepackage{multirow} %para combinar en tablas, usado para la portada
\usepackage{float} %para el float H en tablas
\usepackage{blindtext} % Package to generate dummy text throughout this template 

\renewcommand{\theequation}{\arabic{section}.\arabic{equation}} %definición enumeración de ecuaciones
% para hacer referencia a ecuaciones utilizar \eqref{•}

%reedifiniciónn para hacer enumeración por secciones y subsecciones
\renewcommand{\labelenumi}{\arabic{section}.\arabic{enumi}}
\renewcommand{\labelenumii}{\arabic{section}.\arabic{enumi}.\arabic{enumii}}
\renewcommand{\labelenumiii}{\arabic{section}.\arabic{enumi}.\arabic{enumii}.\arabic{enumiii}}
\renewcommand{\labelenumiv}{\arabic{section}.\arabic{enumi}.\arabic{enumii}.\arabic{enumiii}.\arabic{enumiv}}

\renewcommand{\thefigure}{\arabic{section}.\arabic{figure}} %definición enumeración de ecuaciones
\renewcommand{\thetable}{\arabic{section}.\arabic{table}} %definición enumeración de tablas


%----------New MathCommands by Luis Henríquez    
\newcommand{\der}[2]{\frac{\partial #1}{\partial #2}}
\newcommand{\dero}[2]{\frac{d #1}{d #2}}
\newcommand{\lapl}[4]{\der{}{#1} \p{#2 \der{#3}{#1}}_{#4}}
\newcommand{\p}[1]{\left( #1 \right)} %Parentheses
\newcommand{\ps}[1]{\left[ #1 \right]} %Square
\newcommand{\pc}[1]{\left \{ #1 \right \}} %Curve
\newcommand{\pa}[1]{\left \langle #1 \right \rangle} %Angle


\newcommand{\blue}[1]{{\color{blue}#1}}
\newcommand{\red}[1]{{\color{red}#1}}

\usepackage{etoolbox}
\patchcmd{\thebibliography}{\section*{\refname}}{}{}{}

%\footcite{} se utiliza para agregar citas bibliográficas al píe de pagina y en la pagina de referencias

%-------------------------------FIN BLOQUE DE INSTRUCCIONES
